%-------------------------
% Rezume, a latex resume template for developers
% Author : Nanu Panchamurthy
% Based off of: https://github.com/sb2nov/resume
% License : MIT

% Hope this resume template helps you land an awesome job. If you found this helpful, please consider starring the github repo here, .
%-------------------------



%------------PACKAGES----------------
\documentclass[a4paper,11pt]{article}

\usepackage{verbatim} % reimplements the "verbatim" and "verbatim*" environments

\usepackage{titlesec} % provides an interface to sectioning commands i.e. custom elements

\usepackage{color} % provides both foreground and background color management

\usepackage{enumitem} % provides control over enumerate, itemize and description

\usepackage{fancyhdr} % provides extensive facilities for constructing headers, footers and also controlling their use

\usepackage{tabularx} % defines an environment tabularx, extension of "tabular" with an extra designator x, paragraph like column whose width automatically expands to fill the width of the environment

\usepackage{latexsym} % provides mathematical symbols

\usepackage{marvosym} % provides martin vogel's symbol font which contains various symbols

\usepackage[empty]{fullpage} % sets margins to one inch and removes headers, footers etc..

\usepackage[hidelinks]{hyperref} % removes color and shadow of hyperlinks

\usepackage[normalem]{ulem} % provides "\ul" (uline) command which will break at line breaks

\usepackage[english]{babel} % provides culturally determined typographical rules for wide range of languages
%-----------------------------------------

\input glyphtounicode % converts glyph names to unicode
\pdfgentounicode=1 % ensures pdfs generated are ats readable

%----------FONT OPTIONS-------------------
\usepackage[default]{sourcesanspro} % uses the font source sans pro
\urlstyle{same} % changes url font from default urlfont to font being used by the document
%-----------------------------------------


%----------MARGIN OPTIONS-----------------
\pagestyle{fancy} % set page style to one configured by fancyhdr
\fancyhf{} % clear all header and footer fields

\renewcommand{\headrulewidth}{0in} % sets thickness of linerule under header to zero
\renewcommand{\footrulewidth}{0in} % sets thickness of linerule over footer to zero

\setlength{\tabcolsep}{0in} % sets thickness of column separator in tables to zero

% origin of the document is one inch from the top and from and the left
% oddsidemargin and evensidemargin both refer to the left margin
% right margin is indirectly set using oddsidemargin
\addtolength{\oddsidemargin}{-0.5in}
\addtolength{\topmargin}{-0.5in}

\addtolength{\textwidth}{1.0in} % sets width of text area in the page to one inch
\addtolength{\textheight}{1.0in} % sets height of text area in the page to one inch

\raggedbottom{} % makes all pages the height of current page, no extra vertical space added
\raggedright{} % makes all pages the width of current page, no extra horizontal space added
%------------------------------------------


%--------SECTIONING COMMANDS---------
% \titleformat{<command>}
%   [<shape>]{<format>}{<label>}{<sep>}
%     {<before-code>}[<after-code>]

% command is the sectioning command to be redefined
% shape is the style of the font; scshape stands for small caps style
% format is the format to be applied to whole title- label and text; absent here
% label defines the label
% sep is the horizontal separation between label and title body
% before-code is the code to be executed before
% after-code is the code to be executed after

\titleformat{\section}
  {\scshape\large}{}
    {0em}{\color{blue}}[\color{black}\titlerule\vspace{0pt}]
%-------------------------------------


%--------REDEFINITIONS----------------
% redefines the style of the bullet point
\renewcommand\labelitemii{$\vcenter{\hbox{\tiny$\bullet$}}$}

% redefines the underline depth to 2pt
\renewcommand{\ULdepth}{2pt}
%-------------------------------------


%--------CUSTOM COMMANDS--------------
%\vspace{} defines a vertical space of given size, modifying this in custom commands can help stretch or shrink resume to remove or add content

% resumeItem renders a bullet point
\newcommand{\resumeItem}[1]{
  \item\small{#1}
}

% commands to start and end itemization of resumeItem, rightmargin set to 0.11in to avoid the overflow of resumetItem beyond whatever resumeItemHeading is being used
\newcommand{\resumeItemListStart}{\begin{itemize}[rightmargin=0.11in]}
\newcommand{\resumeItemListEnd}{\end{itemize}}

% resumeSectionType renders a bolded type to be used under a section, used as skill type here, middle element is used to keep ":"s in the same vertical line
\newcommand{\resumeSectionType}[3]{
  \item\begin{tabular*}{0.96\textwidth}[t]{
    p{0.2\linewidth}p{0.02\linewidth}p{0.81\linewidth}
  }
    \textbf{#1} & #2 & #3
  \end{tabular*}\vspace{-2pt}
}

% resumeTrioHeading renders three elements in three columns with second element being italicized and first element bolded, can be used for projects with three elements
\newcommand{\resumeTrioHeading}[3]{
  \item\small{
    \begin{tabular*}{0.96\textwidth}[t]{
      l@{\extracolsep{\fill}}c@{\extracolsep{\fill}}r
    }
      \textbf{#1} & \textit{#2} & #3
    \end{tabular*}
  }
}

% resumeQuadHeading renders four elements in a two columns with the second row being italicized and first element of first row bolded, can be used for experience and projects with four elements
\newcommand{\resumeQuadHeading}[4]{
  \item
  \begin{tabular*}{0.96\textwidth}[t]{l@{\extracolsep{\fill}}r}
    \textbf{#1} & #2 \\
    \textit{\small#3} & \textit{\small #4} \\
  \end{tabular*}
}

% resumeQuadHeadingChild renders the second row of resumeQuadHeading, can be used for experience if different roles in the same company need to added
\newcommand{\resumeQuadHeadingChild}[2]{
  \item
  \begin{tabular*}{0.96\textwidth}[t]{l@{\extracolsep{\fill}}r}
    \textbf{\small#1} & {\small#2} \\
  \end{tabular*}
}

% commands to start and end itemization of resumeQuadHeading, lefmargin for left indent of 0.15in for resumeItems
\newcommand{\resumeHeadingListStart}{
  \begin{itemize}[leftmargin=0.15in, label={}]
}
\newcommand{\resumeHeadingListEnd}{\end{itemize}}
%-------------------------------------------


%__________________RESUME____________________
% You can rearrange sections in any order you may prefer
\begin{document}

%-----------CONTACT DETAILS------------------
% Make sure all the details are correct, you can add more links in the first row of second column if needed

\begin{tabular*}{\textwidth}{l@{\extracolsep{\fill}}r}
  \textbf{\Large Matheus Henrique Sant'Anna Cardoso \vspace{2pt}} & % row = 1, col = 1
  \textbf{Location}: Rio de Janeiro, RJ, Brazil \\ % row = 1, col = 2
  \href{https://www.linkedin.com/in/mhscardoso/}{\uline{LinkedIn}} $|$ % row = 2, col = 1
  \href{https://github.com/mhscardoso}{\uline{GitHub}} &
  \textbf{Email}: \href{mailto:mhcardoso2001@gmail.com}{\uline{mhcardoso2001@gmail.com}}\vspace{4pt} \\% row = 2, col = 2
  & \textbf{Mobile}: +55 21 988307754 \\ % row = 2, col = 2
\end{tabular*}
%--------------------------------------------


%-----------SUMMARY--------------------------
% Keep this short, simple and straigth to point

\section{Profile}
\small{
    I'm a backend developer with over \textbf{1 year experience in Python} and over \textbf{6 month experience in TypeScript} working with popular frameworks such as \textbf{Flask, Django and NestJS}. I also have a little experience with \textbf{Docker}.

    I have the \textbf{ability to learn development tools} and the \textbf{habit of reading APIs, frameworks and libraries documentation}.
}
%--------------------------------------------


%--------------SKILLS------------------------
% Add or remove resumeSectionTypes according to your needs

\section{Technical Skills}
  \resumeHeadingListStart{}
    \resumeSectionType{Languages}{:}{Python, C, Go, JavaScript, TypeScript}
    \resumeSectionType{Markup Languages}{:}{HTML, CSS, LaTeX}
    \resumeSectionType{Frameworks}{:}{Flask, Django, NestJS}
    \resumeSectionType{Databases}{:}{PostgreSQL, SQLServer}
    
  \resumeHeadingListEnd{}
%--------------------------------------------


%-----------EXPERIENCE-----------------------
% Distill all your talking points to small bullet points which follow the pattern "challenge-action-result" for maximum efficiency. Try to quantify (use numbers) your points whenver possible, highlist words of importance

\section{Experience}
\resumeHeadingListStart{}

   \resumeQuadHeading{Backend Developer (Consultant)}{Feb 2022 -- Apr 2023}
  {Fluxo Consultoria (Junior Enterprise)}{Rio de Janeiro, RJ, Brazil}

    \resumeItemListStart{}

      \resumeItem{Worked with Flask and Django to create \textbf{REST APIs} for websites, dealing with PostgreSQL as database and lagacy code}

      \resumeItem{Worked with the backend team to implement the \textbf{NestJS} framework in the company}

    \resumeItemListEnd{}
  
    \resumeQuadHeading{IT Projects Intern}{Jul 2023 -- Now}
    {ALM Seguradora S/A}{Rio de Janeiro, RJ, Brazil}
  
      \resumeItemListStart{}
  
        \resumeItem{Project management for the implementation of new technology for the insurance company, getting involved with IT companies to improve the efficiency of processes.}
  
        \resumeItem{Process mapping with the company's employees.}
  
      \resumeItemListEnd{}

\resumeHeadingListEnd{}
%---------------------------------------------


%-----------EDUCATION-------------------------
% Mention your CGPA, if its good, in the first row of second column

\section{Education}
  \resumeHeadingListStart{}

    \resumeQuadHeading{Federal University of Rio de Janeiro}{Rio de Janeiro, RJ, Brazil}
    {Computer and Information Engineering}{May 2021 -- Jun 2026 (Forecast)}

    \resumeItemListStart{}

    \resumeItem{At University I had the opportunity to improve my skills in OOP in Python and Data Science tools such as Pandas and Numpy}

    \resumeItem{I also worked with LaTeX for the production of homework and reports}

    \resumeItemListEnd{}

  \resumeHeadingListEnd{}
%---------------------------------------------


%-----------PROJECTS--------------------------
% Use resumeQuadHeading if four elements are feasible (ex: demo video link), else use resumeTrioHeading. Keep the bullet points simple and concise and try to cover wide variety of skills you have used to build these projects

\section{Projects}
  \resumeHeadingListStart{}

    \resumeTrioHeading{Finance}{Python, OOP, SQLite3, Git}{\href{https://github.com/mhscardoso/finance}{\uline{Source Code}}}
      \resumeItemListStart{}
        \resumeItem{Python desktop application to simulate buying and selling of shares}
        \resumeItem{Working directly with \textbf{SQL Queries}}
        \resumeItem{Working with \textbf{OOP} concepts}
      \resumeItemListEnd{}

      \resumeTrioHeading{Wisard}{C, Git, AI}{\href{https://github.com/mhscardoso/Wisard_IA}{\uline{Source Code}}}
      \resumeItemListStart{}
        \resumeItem{A simple AI written in C for the Algorithms and Programming course.}
      \resumeItemListEnd{}
  \resumeHeadingListEnd{}
%--------------------------------------------


%----------------OTHERS----------------------
% You can add your acheivements, accolades, certifications etc. here.
\section{Languages}
  \resumeHeadingListStart{}
  \resumeSectionType{Portuguese}{:}{Native Language}
  \resumeSectionType{English}{:}{Intermediary}
  \resumeHeadingListEnd{}

%--------------------------------------------

\end{document}